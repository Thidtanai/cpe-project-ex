\chapter{\ifenglish Conclusions and Discussions\else บทสรุปและข้อเสนอแนะ\fi}

\section{\ifenglish Conclusions\else สรุปผล\fi}

\hspace{4ex} 
จากการพัฒนาโครงงาน Project Matching Management Platform ระบบที่ได้ สามารถทำงานได้ตามวัตถุประสงค์ของโครงงานซึ่งก็คือสามารถพัฒนาเว็บแอปพลิเคชันที่สามารถรองรับการประกาศกิจกรรมและเข้าร่วมกิจกรรมได้อย่างถูกต้อง อีกทั้งสามารถพัฒนาระบบที่สามารถนำข้อมูลของผู้ใช้มาใช้ในการแนะนำกิจกรรมให้กับผู้ใช้ได้

\section{\ifenglish Challenges\else ปัญหาที่พบและแนวทางการแก้ไข\fi}

ในการทำโครงงานนี้ พบว่าเกิดปัญหาหลักๆ ดังนี้
\begin{enumerate}
    \item การพูดคุยระหว่างผู้พัฒนามีจำนวนน้อยเกินไปทำให้งานดำเนินไปได้ด้วยความล่าช้า และทำให้เกิดปัญหาในการดำเนินงาน
    \item ไม่สามารถหาข้อมูลสถิติการเข้าร่วมกิจกรรมต่างๆของนักศึกษาแบ่งตามแท็กกิจกรรมได้ ทำให้ระบบแนะนำกิจกรรมที่ได้มีประสิทธิภาพไม่เป็นอย่างที่ออกแบบไว้
    \item คิดออกแบบโครงงานได้ไม่ดีเพียงพอ ทำให้ตอนพัฒนาจริงต้องปรับเปลี่ยนอะไรต่างๆจากที่ออกแบบไว้มากกว่าที่ควร
    \item วางแผนเวลาการดำเนินงานได้ไม่ดีพอ และ ใช้เวลาในการคิดออกแบบระบบต่างๆมากเกินไป แล้วไม่ได้ลงมือปฏิบัติจริงซักที ทำให้เวลาในการพัฒนาไม่เพียงพอ
\end{enumerate}

\section{\ifenglish%
Suggestions and further improvements
\else%
ข้อเสนอแนะและแนวทางการพัฒนาต่อ
\fi
}

ข้อเสนอแนะเพื่อพัฒนาโครงงานนี้ต่อไป มีดังนี้

\begin{enumerate}
    \item พัฒนาระบบแนะนำจากการนำข้อมูลของแท็กของผู้ใช้ไปทำการเสิร์ช เป็นการนำข้อมูลต่างๆของผู้ใช้ไปทำการประมวลผลหาแนวโน้มร่วมกับผู้ใช้อื่นๆด้วย
    \item จัดการเคลียร์โค้ดใหม่ให้มีประสิทธิภาพมากยิ่งขึ้น 
    \item พัฒนาระบบให้สามารถใช้กับ CMU Account ได้
    \item เพิ่มระบบการจัดการสมาชิกในกิจกรรม
    \item เพิ่มระบบรีวิวกิจกรรม
    \item เพิ่มการทำงานฝั่ง Admin
    \item เพิ่มระบบสรุปผลการเข้าร่วมกิจกรรมของผู้ใช้
    \item นำเว็บแอปพลิเคชันไป deploy บน Cloud
\end{enumerate}
