\chapter{\ifenglish Introduction\else บทนำ\fi}

\section{\ifenglish Project rationale\else ที่มาของโครงงาน\fi}
บุคลากรภายในมหาวิทยาลัยเชียงใหม่จำนวนมากต้องการที่จะหาคนมาเข้าร่วมงานอีเว้นท์ หรือ เข้ามาช่วยในการจัดกิจกรรมต่างๆ เเต่ไม่สามารถหาผู้เข้าร่วมได้ ซึ่งในหลายๆครั้งนั้น อาจมีนักศึกษาหรือบุคลากรจำนวนมากที่สนใจเข้าร่วมเเต่ไม่ได้เข้าร่วมเพียงเพราะไม่ได้รับข่าวสารการประกาศ ซึ่งเป็นเพราะช่องทางที่ผู้จัดประกาศนั้นเข้าไม่ถึงบุคลากรเหล่านั้นด้วยเหตุผลต่างๆ เช่น ประกาศในเครือข่ายสังคมออนไลน์(Social Network) เเล้วผู้ที่สนใจไม่เห็นเนื่องด้วยอาจไม่ได้ติดตามช่องทางที่ประกาศหรืออาจเพราะถูกบดบังด้วยอัลกอริทึม(Algorithm) ของเครือข่ายสังคมออนไลน์นั้นๆ

จากปัญหาดังกล่าว ผู้จัดทำโครงงานจึงได้คิดที่จะพัฒนาเว็บแอปพลิเคชั่น(Web Application) ที่เป็นศูนย์กลางในการสร้างและประกาศกิจกรรมเพื่อแก้ปัญหาเกี่ยวกับการประกาศกิจกรรม รวมทั้งยังมีระบบการรับสมัครทีมงานเพื่อช่วยให้การจัดกิจกรรมเป็นไปอย่างราบรื่นมากขึ้น


\section{\ifenglish Objectives\else วัตถุประสงค์ของโครงงาน\fi}
\begin{enumerate}
    \item พัฒนาเว็บแอปพลิเคชันที่สามารถรองรับการประกาศกิจกรรม และผู้ที่สนใจเข้าร่วมกิจกรรมสามารถเข้ามามีส่วนร่วมกันได้อย่างถูกต้อง
    \item พัฒนาเว็บแอปพลิเคชันที่นำความรู้ทางด้านการวิเคราะห์ข้อมูล(Data Analytic) มาใช้ในการเเนะนำกิจกรรมให้เเก่ผู้ใช้ โดยกิจกรรมที่เเนะนำจะต้องสอดคล้องกับความสนใจของผู้ใช้รายนั้นๆ
\end{enumerate}

\section{\ifenglish Project scope\else ขอบเขตของโครงงาน\fi}

\subsection{\ifenglish Hardware scope\else ขอบเขตด้านฮาร์ดแวร์\fi}
\begin{enumerate}
    \item คอมพิวเตอร์เพื่อใช้พัฒนาเว็บแอปพลิเคชันและตรวจสอบผลลัพธ์ผ่านเว็บบราวเซอร์(Web browser)
    \item สมาร์ทโฟน(Smartphone)ระบบแอนดรอยด์(Android) เพื่อใช้ตรวจสอบผลลัพธ์ผ่านเว็บบราวเซอร์
\end{enumerate}

\subsection{\ifenglish Software scope\else ขอบเขตด้านซอฟต์แวร์\fi}
\begin{enumerate}
    \item การเข้าถึงเว็บแอปพลิเคชัน สามารถเข้าผ่านเว็บบราวเซอร์ต่างๆ เช่น Chrome, Firefox เป็นต้น
    \item ส่วนบัญชีผู้ใช้ คือการยืนยันตัวตนผ่าน CMU-OAuth
    \item ส่วนแสดงกิจกรรมทั้งหมด ผู้ใช้จะสามารถดูข้อมูลเบื้องต้นของกิจกรรมต่างๆได้ โดยในส่วนนี้จะแบ่งเป็นกิจกรรมที่มีผู้สนใจเยอะ และกิจกรรมทั้งหมด
    \item ส่วนการสร้างกิจกรรม ผู้ใช้สามารถสร้างกิจกรรมใหม่ขึ้นมา โดยระบุรายละเอียดต่างๆของกิจกรรม สามารถเลือกได้ว่าจะต้องการผู้สมัครเข้าร่วมกิจกรรมหรือไม่
    \item ส่วนแสดงกิจกรรมเฉพาะ เมื่อผู้ใช้เข้ามาส่วนนี้ ผู้ใช้จะสามารถดูข้อมูลของกิจกรรมได้โดยละเอียด และสามารถสมัครเข้าร่วมกิจกรรมได้
    \item ส่วนตอบรับการเข้าร่วมกิจกรรม ผู้ที่สร้างกิจกรรมสามารถเลือกได้ว่าจะให้ผู้สมัครคนไหนมีสิทธิเข้าร่วมกิจกรรมบ้าง
    \item ส่วนการให้คะแนนกิจกรรม ผู้เข้าร่วมกิจกรรมสามารถให้คะแนนกิจกรรมและผู้จัดได้ ส่วนผู้จัดก็สามารถให้คะแนนผู้เข้าร่วมได้เช่นกัน
    \item ส่วนแดชบอร์ด ผู้ใช้สามารถดูสถิติต่างๆที่ตนเองสนใจได้ เช่น ผู้สร้างกิจกรรมสามารถดูผลตอบรับของผู้ใช้คนอื่นๆ, ผู้ดูแลระบบสามารถดูสถิติโดยรวมของเว็บแอปพลิเคชันได้ เป็นต้น
\end{enumerate}

\subsection{ขอบเขตด้านกลุ่มผู้ใช้}
นักศึกษาและบุคลากรของมหาวิทยาลัยเชียงใหม่ที่มี CMU-Account

\subsection{ขอบเขตด้านข้อมูล}
\begin{enumerate}
    \item กิจกรรมประเภทต่างๆ เช่น รับน้องขึ้นดอย, CPE Music box, จับกลุ่มออกกำลังกาย เป็นต้น
    \item ข้อมูลของผู้ใช้ที่ได้รับจาก CMU-Account
\end{enumerate}

\section{\ifenglish Expected outcomes\else ประโยชน์ที่ได้รับ\fi}
\begin{enumerate}
    \item สามารถทำให้กิจกรรมต่างๆที่มาฝากประกาศในช่องทางเรา เข้าถึงกลุ่มเป้าหมายได้มากขึ้น
    \item สามารถทำให้นักศึกษาและบุคลากรในมหาวิทยาลัย ได้เห็นกิจกรรมที่ตัวเองสนใจได้ง่ายขึ้น
    \item สามารถทำให้การหาข้อมูลกิจกรรมต่างๆนั้น สะดวกมากยิ่งขึ้น
\end{enumerate}

\section{\ifenglish Technology and tools\else เทคโนโลยีและเครื่องมือที่ใช้\fi}

\subsection{\ifenglish Hardware technology\else เทคโนโลยีด้านฮาร์ดแวร์\fi}
\begin{enumerate}
    \item ASUS Vivobook Pro 15 : สำหรับพัฒนาเว็บแอปพลิเคชัน
    \item Huawei P20 Pro : สำหรับตรวจสอบการแสดงผลบนสมาร์ทโฟน
    \item Asus Vivobook x509JP : สำหรับพัฒนาเว็บแอปพลิเคชัน
    \item Realme x7 Pro 5G : สำหรับตรวจสอบการแสดงผลบนสมาร์ทโฟน
    \item Apple iPad 7 wifi : สำหรับตรวจสอบการแสดงผลบนเเท็บเล็ต
\end{enumerate}
\subsection{\ifenglish Software technology\else เทคโนโลยีด้านซอฟต์แวร์\fi}
\begin{enumerate}
    \item Figma : เว็บแอปพลิเคชันที่ใช้ในการออกแบบ User Interface ของเว็บไซต์
    \item Jira Software : เว็บแอปพลิเคชันที่ใช้ในการวางแผนงาน, แบ่งงาน และดูความคืบหน้าของแต่ละงาน
    \item GitHub : Version control ที่สามารถเก็บไฟล์ได้บนอินเทอร์เน็ต
    \item Visual Studio Code : โปรแกรมที่ใช้ในการเขียนโค้ด โดยมีจุดเด่นคือมีส่วนขยายโปรแกรมที่สร้างโดยผู้ใช้ทั่วโลก
    \item React : Javascript Library ที่ช่วยในการสร้าง User interface
    \item TypeScript : ภาษาโปรแกรมที่พัฒนาต่อมาจาก Javascript โดยเพิ่ม Static typing เพื่อตรวจสอบความผิดพลาดของโปรแกรมได้โดยง่าย
    \item Node.js : Node.js คือ JavaScript runtime สำหรับฝั่ง Server และเป็น Open Source ซึ่งเขียนด้วยภาษา JavaScript ใช้สำหรับเป็น Web Server
    \item MongoDB : ฐานข้อมูลประเภท NoSQL ที่ใช้ในการจัดเก็บข้อมูลของผู้ใช้และกิจกรรมต่างๆ
    \item MongoDB Compass : เป็นเครื่องมือในการจัดการข้อมูล database ของ MongoDB ในรูปแบบ GUI ในการอํานวยความสะดวกในการ วิเคราะห์ข้อมูล การดําเนินการกับข้อมูล (CRUD)
    \item Postman : เป็นเครื่องมือที่ใช้สำหรับพัฒนาและทดสอบ API 
\end{enumerate}
\section{\ifenglish Project plan\else แผนการดำเนินงาน\fi}

\begin{plan}{10}{2023}{3}{2024}
    % \planitem{6}{2023}{6}{2023}{ค้นหาหัวข้อที่สนใจและอาจารย์ที่ปรึกษา}
    % \planitem{6}{2023}{8}{2023}{ค้นหาข้อมูล ทฤษฎีที่เกี่ยวข้องและกำหนดขอบเขต}
    % \planitem{7}{2023}{8}{2023}{ออกแบบ Mockup คร่าวๆของเว็บด้วย Figma}
    % \planitem{8}{2023}{8}{2023}{ออกแบบ Diagram ของระบบแบบคร่าวๆ}
    % \planitem{8}{2023}{9}{2023}{หาข้อมูลเกี่ยวกับกิจกรรมตัวอย่าง}
    % \planitem{8}{2023}{10}{2023}{ออกแบบ Flow ของระบบ}
    % \planitem{8}{2023}{10}{2023}{ออกแบบ UX/UI ของเว็บด้วย Figma}
    % \planitem{9}{2023}{10}{2023}{เขียนรายงานและนำเสนอ 261491}
    \planitem{10}{2023}{10}{2023}{ศึกษา Algorithm สำหรับระบบ Recommendation}
    % \planitem{10}{2023}{10}{2023}{ศึกษาการทำ Data Visualization สำหรับหน้าแดชบอร์ด}
    \planitem{10}{2023}{11}{2023}{ออกแบบฐานข้อมูล}
    \planitem{11}{2023}{2}{2024}{พัฒนาเว็บแอปพลิเคชัน}
    \planitem{11}{2023}{2}{2024}{ทดสอบกับข้อมูลจำลองและปรับปรุงระบบ}
    \planitem{1}{2024}{3}{2024}{เขียนรายงานและนำเสนอ 261492}

\end{plan}

\section{\ifenglish Roles and responsibilities\else บทบาทและความรับผิดชอบ\fi}
\begin{enumerate}
    \item ส่วนที่ทำงานร่วมกันได้แก่ การวางแผนงาน, การค้นหาความรู้และทฤษฎีที่เกี่ยวข้อง
    \item ส่วนที่รับผิดชอบโดยนาย ณัฏฐพล ตันจอ 620610786 ได้แก่ การทำส่วนติดต่อผู้ใช้ของเว็บแอปพลิเคชัน (Front-End)
    \item ส่วนที่รับผิดชอบโดยนาย นายธิษณ์ธนัย แก้วเพ็ชร์ 630610741 ได้แก่ การออกแบบหน้าเว็บแอปพลิเคชัน, พัฒนาส่วน back-end กับ database และระบบ Recommendation
\end{enumerate}

\section{\ifenglish%
Impacts of this project on society, health, safety, legal, and cultural issues
\else%
ผลกระทบด้านสังคม สุขภาพ ความปลอดภัย กฎหมาย และวัฒนธรรม
\fi}

โครงงานนี้จะช่วยเพิ่มช่องทางการติดตามงานกิจกรรมต่างๆภายในมหาวิทยาลัยเชียงใหม่ให้สามารถเข้าถึงนักศึกษาเเละบุคลากรได้มากขึ้นทำให้จำนวนผู้เข้าร่วมมีโอกาสสูงขึ้น ซึ่งจะช่วยส่งเสริมการมีส่วนร่วมทางสังคมของบุคลากรภายในมหาวิทยาลัยเชียงใหม่ได้